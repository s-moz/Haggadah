% This file defines the signposting text to be inserted into haggadah.tex  In this way
% multiple formats can be typeset very quickly.  haggadah.tex should not normally
% need changing.  Note that this is not the most readable way to insert text
% into a LaTeX document, but it is the most powerful: the macros defined here
% are directly excecuted when building the document.

%\usepackage{fontspec}
%\usepackage{polyglossia}
%\setdefaultlanguage{english}
%\setotherlanguages{czech, hebrew}
%\newfontfamily{\hebrewfont}{New Peninim MT}

\usepackage{multicol}
\usepackage{changepage}
\usepackage{enumitem} %for the manishtana
% For the title page
\newcommand{\contents}{}
%
\newcommand{\contentsTranslation}{}
% 
\newcommand{\Neirot}{
	{\noindent\color{midblue}We normally begin after Havdalah by lighting the festival candles, but today we will come back to this once the sun has set fully, so we can sleep on time.}
	\CandlesBlessingHeb
	\par
	\textit{\CandlesBlessingTlit}
	\begin{multicols}{2}
		\CandlesBlessingCz
		
		\columnbreak 
		\CandlesBlessingEng
	\end{multicols}
	}

\newcommand{\Kaddeish}{
	\WineBlessingHeb
	\textit{\WineBlessingTlit}
	\begin{multicols}{2}
		\WineBlessingCz
		
		\columnbreak 
		\WineBlessingEng
	\end{multicols}
}

\newcommand{\Shehechyanu}{
	\ShehechyanuBlessingHeb
	\textit{\ShehechyanuBlessingTlit}
	\par
		\begin{multicols}{2}
		\ShehechyanuBlessingCz
		
		\columnbreak \ShehechyanuBlessingEng
		\end{multicols}}

\newcommand{\urchatz}{
	\textit{Druhým bodem večeře je Urchac – rituální očista. }	\textit{Pour water on each of your hands three times, alternating between your
	hands}}

 
\newcommand{\Karpas}{
{\noindent\color{midblue}We dip Parsley into salt water to symbolise the sweat and tears of the slaves}
\KarpasBlessingHeb
\textit{\KarpasBlessingTlit}
\begin{multicols}{2}
	\KarpasBlessingCz
	
	\columnbreak 
	\KarpasBlessingEng
\end{multicols}
}
\newcommand{\Yachatz}{%
\noindent{rozlome střední maces (levi), jednu polovinu z něho odloží jako zákusek (afikoman) na závěr pesachové hostiny a zbývající polovinu vrátí na její místo.}
{\noindent\color{midblue}We break the middle of 3 matzah and replace the smaller half, as all three are lifted and the following Aramaic passage is read to commemorate the hardship felt by the israelites in egypt:}
\begin{multicols}{2}
	\HaLachmaTlit
	
	\columnbreak \HaLachmaEng
\end{multicols}
\textit{Now the youngest participant wraps up and hides the larger piece as the Afikomen ("dessert") for the oldest participant to find later} 
}
%
\newcommand{\MaNishtana}{%
\textit{\noindent\color{midblue}We take turns to read these questions}
\MaNishtanaHeb
\MaNishtanaTlit
\MaNishtanaEng
\MaNishtanaCz
	\begin{enumerate}
		\item \MaNishtanaHebi \\ \MaNishtanaTliti \\\MaNishtanaEngi \\ \MaNishtanaCzi \\
		\item \MaNishtanaHebii \\ \MaNishtanaTlitii \\\MaNishtanaEngii \\ \MaNishtanaCzii \\
		\item \MaNishtanaHebiii\\ \MaNishtanaTlitiii \\\MaNishtanaEngiii \\ \MaNishtanaCziii \\
		\item \MaNishtanaHebiv \\ \MaNishtanaTlitiv \\\MaNishtanaEngiv \\ \MaNishtanaCziv \\
		
	\end{enumerate}

\newcommand{\MaNishtanaAnswerCz}
\newcommand{\MaNishtanaAnswerEng}}

\newcommand{\MaggidIntro}{%
	\centering\textit{Pour the second glass. Do not drink it yet.}\\
	\MaggidText
	\\
	\par
	\centering\textit{Raise the glass and read together:}\\
	\par
	\PromiseToast 
	\\
	\centering\textit{now put the glass down without drinking}\\
	%

}
\newcommand{\Plagues}{
	{\noindent\color{midblue}{G-d brought about 10 Plagues upon the Egyptians in Egypt:\\}
		\textit{Deset ran přivedl Svatý – buď požehnán – na Egypťany v Egyptě:}}
	image of plagues
	As we read the name of each plague we spill a drop of drink on each one, like a tear for the suffering they caused in the world.
	\begin{adjustwidth}{15mm}{15mm}
		\begin{multicols}{3}
			\PlaguesTextCz
		
			\columnbreak 
			\PlaguesTextEng
			
			\columnbreak
			\PlaguesTextHeb
		\end{multicols}
			\lipsum[2]
	\end{adjustwidth}
}

\newcommand{\Dayyeinu}{
	%\textit{
	%	Persian Jews hit each other with a Spring Onion at each of the following lines, to symbolise the whips of the slavemasters.\\
	%	{\large Sarah is allergic to onion so is exempt from being whacked}
	%	}
	
	*maybe find a song version of the dayyeinu and put the words out so we can listen and hit when we spot the word dayyeinu*\\
	\DayyeinuIntro
	\DayyeinuText
	\vspace*{2ex}
	\RabbiGamliel
	\vspace*{2ex}
	\EveryGenerationText
	
	We remember how G-d redeemed our ancestors from Egypt, and enabled us to spend this Pesach Seder together. May we arrive at future holidays in peace and happiness.
	
	\WineBlessingHeb
	\WineBlessingTlit
	\WineBlessingEng
	
	\textit{Opřeni o levý loket vyprázdníme druhý pohár.}
	Now we drink!}
%
\newcommand{\Handwashing}{
\textit{pour water three times on your right hand then three times on your left hand and then read:}
\NetilatYadayim}
%
\newcommand{\HaMotzi}{
	\textit{We bless the unleavened bread just as we bless any bread at the start of a meal:}
	\HaMotziBlessingHeb
	\textit{\HaMotziBlessingTlit}
	\begin{multicols}{2}
		\HaMotziBlessingCz
		
		\columnbreak \HaMotziBlessingEng
	\end{multicols}
}
\newcommand{\Matzah}{
	\textit{Vypravěč ukáže stolovníkům macesy a říká:} 
	Z jakého důvodu jíme tuto maca? Proto, že než se našim předkům 
	zjevil Svatý, buď požehnán, a vykoupil je, nestačilo jim vykynout těsto. Je 
	přec řečeno: „Z těsta, které vynesli z Egypta a které nezkvasilo, pekli 
	macesové placky, jelikož byli z Egypta vyhnáni a nemohli se zdržovat a 
	neudělali si ani zásoby na cestu.“ 
\textit{We recite the blessing over Matzah because consuming it is a specific Torah command:}
\MatzahBlessingHeb
\textit{\MatzahBlessingTlit}
	\begin{multicols}{2}
	\MatzahBlessingCz
	
	\columnbreak \MatzahBlessingEng
\end{multicols}
}
%
\newcommand{\Maror}{
 In creating a holiday about the joy of freedom, we turn the story of our bitter
history into a sweet celebration. \\ We recognize this by dipping our bitter horseradish (křen) into the sweet charoset. Try to taste both the flavours at once.\\
\vspace*{2ex}
\textit{Vypravěč ukáže maror a říká:}
Proč jíme tuto hořkost (maror)? Proto, že Egypťané ztrpčovali život 
našich předků v Egyptě, jak je řečeno: „Ztrpčovali jejich život dřinou s 
hlínou a cihlami a vší tou prací na poli a všemi dalšími pracemi, jimiž jim 
otročili při nucených pracích.“\\
\vspace*{2ex}
\textit{namáčíme náš křen do charosetu}\\
\MarorBlessingHeb
\textit{\MarorBlessingTlit}
\begin{multicols}{2}
	\MarorBlessingCz
	
	\columnbreak \MarorBlessingEng
\end{multicols}
}
%
\newcommand{\Koreich}{
	%\includegraphics{Koreich}
	\KoreichText}

\newcommand{\Bareich}{
\BareichEng}

\newcommand{\Hallel}{%
  {\HallelHeb
  	\begin{multicols}{2}
  		\HallelTlit
  		\columnbreak \HallelEng
  	\end{multicols}
}}

\newcommand{\Elijah}{
	\ElijahText
	\StrangersText
	\par \textit{We sing this song before closing the door:}
	\EliahuHaNavi}

\newcommand{\Miriam}{
\Miriamreading
\MiriamBlessing}

\newcommand{\Nirtzah}{
	As we conclude our Seder, we think about the following year with all its ups, downs and opportunities, and consider how we can bring sweetness to any bitterness we encounter, how we can give and be held by community, and how we will change and grow while retaining the promises, responsibilities and blessings of being Jewish. And either in a literal or a symbolic sense, where Jerusalem signifies a hope for a warm, loving and peaceful community of friends and family, we all raise a glass and toast together: \par
	\NirtzahBlessing}
% File paths: we don't use symlinks as (a) not all platforms support them, and
% (b) they don't fit nicely with the flow we're using.

%\newcommand{\kyriePath}{../Ordinaries/masses/4/kyrie}
%\newcommand{\gloriaPath}{../Ordinaries/masses/4/gloria}
%\newcommand{\sanctusPath}{../Ordinaries/masses/4/sanctus}
%\newcommand{\agnusPath}{../Ordinaries/masses/4/agnus}
%\newcommand{\itePath}{../Ordinaries/masses/4/ite}
% 
%\newcommand{\creedPath}{../Ordinaries/credo/1/credo}
% 
%%% Local Variables:
%%% mode: latex
%%% TeX-master: "missalette"
%%% End:
