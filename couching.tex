% This file defines the signposting text to be inserted into haggadah.tex  In this way
% multiple formats can be typeset very quickly.  haggadah.tex should not normally
% need changing.  Note that this is not the most readable way to insert text
% into a LaTeX document, but it is the most powerful: the macros defined here
% are directly excecuted when building the document.

%\usepackage{fontspec}
%\usepackage{polyglossia}
%\setdefaultlanguage{english}
%\setotherlanguages{czech, hebrew}
%\newfontfamily{\hebrewfont}{New Peninim MT}

\usepackage{multicol}
%\usepackage{paracols}
\usepackage{setspace}
\usepackage{microtype}
\usepackage{changepage}
\usepackage{ragged2e}
\usepackage{enumitem} %for the manishtana
% For the title page
\newcommand{\contents}{}
%
\newcommand{\contentsTranslation}{}
% 
\newcommand{\Neirot}{
	{\noindent \textit{Protože je šabat, přeskočíme tuto část, dokud slunce nezapadne.}}
	\\{\noindent\color{midblue} \textit{We normally begin after Havdalah by lighting the festival candles, but today we will start the Seder before sunset so we finish in time to sleep.}}
	\CandlesBlessingHeb
	\textit{\CandlesBlessingTlit}
	\begin{multicols}{2}
		\CandlesBlessingCz
		
		\columnbreak 
		\CandlesBlessingEng
	\end{multicols}
	}

\newcommand{\Kaddeish}{
	\WineBlessingHeb
	\textit{\WineBlessingTlit}
	\begin{multicols}{2}
		\WineBlessingCz
		
		\columnbreak 
		\WineBlessingEng
	\end{multicols}
}

\newcommand{\Shehechyanu}{
	\ShehechyanuBlessingHeb
	\textit{\ShehechyanuBlessingTlit}
	\par
		\begin{multicols}{2}
		\ShehechyanuBlessingCz
		
		\columnbreak \ShehechyanuBlessingEng
		\end{multicols}}

\newcommand{\urchatz}{
	\textit{Druhým bodem večeře je Urchac – rituální očista. } \\	\textit{ \color{midblue} Pour water on each of your hands three times, alternating between your
	hands}}

 
\newcommand{\Karpas}{
{\noindent\textit{Petržel namáčíme do slané vody, aby symbolizovala pot a slzy otroků}\\
	\noindent\color{midblue} \textit{We dip Parsley into salt water to symbolise the sweat and tears of the slaves}}
\KarpasBlessingHeb
\textit{\KarpasBlessingTlit}
\begin{multicols}{2}
	\KarpasBlessingCz
	
	\columnbreak 
	\KarpasBlessingEng
\end{multicols}
}
\newcommand{\Yachatz}{%
\noindent \textit{\color{black} Rozlome střední maces (levi), jednu polovinu z něho odloží jako zákusek (afikoman) na závěr pesachové hostiny a zbývající polovinu vrátí na její místo.}\\
{\noindent\color{midblue} \textit{We break the middle of 3 matzah and replace the smaller half, as all three are lifted and the following Aramaic passage is read to commemorate the hardship felt by the israelites in egypt:}} \color{black}
\begin{multicols}{2}
	\HaLachmaTlit
	
	\columnbreak \HaLachmaEng
\end{multicols}
\textit{\noindent Nejmladší zabalí a schová afikomana, aby ho nejstarší našel po jídle. \\ {\color{midblue} Now the youngest participant wraps up and hides the larger piece as the Afikomen ("dessert") for the oldest participant to find later.}} 
}
%
\newcommand{\MaNishtana}{
	{\onehalfspacing
	\centering \color{black}Ty čtyři otázky z Talmudu\\
	\textit{\noindent\color{midblue}We take turns to read these questions}\par
	{\color{darkblue} \MaNishtanaHeb \\
	\textit{\MaNishtanaTlit} \\
	{\color{midblue} \MaNishtanaEng} \\
	{\color{black} \MaNishtanaCz}\\ 
	\par
	\vspace*{4ex}
	\\
	\LARGE 1. \\ \raggedleft 
	\vspace*{2ex}
	\MaNishtanaHebi \\
	\vspace*{1mm}
	\textit{\MaNishtanaTliti} \\ {\color{midblue} \MaNishtanaEngi}\\ {\textcolor{black} \MaNishtanaCzi} \\ \raggedright 
	\vspace*{2ex}
	\MaNishtanaHebib \\
	\vspace*{1mm}
	\textit{\MaNishtanaTlitib} \\{\color{midblue} \MaNishtanaEngib} \\ {\color{black} \MaNishtanaCzib} \\ \raggedright
	\par
	\\ \centering
	\LARGE 2.\\ \raggedleft 
	\vspace*{1mm}
	\MaNishtanaHebii \\ 
	\vspace*{1mm}
	\textit{\MaNishtanaTlitii} \\ {\color{midblue} \MaNishtanaEngii} \\ {\textcolor{black} \MaNishtanaCzii} \\ \raggedright 
	\vspace*{2ex}
	\MaNishtanaHebiib \\ 
	\vspace*{1mm}
	\textit{\MaNishtanaTlitiib} \\ {\color{midblue} \MaNishtanaEngiib} \\ {\color{black} \MaNishtanaCziib} \\
	\raggedright 
	\par
	\\ \centering
	\LARGE 3.\\ \raggedleft 
	\vspace*{1mm}
	\MaNishtanaHebiii \\
	\vspace*{1mm}
	\textit{\MaNishtanaTlitiii} \\{\color{midblue} \MaNishtanaEngiii} \\ {\color{black}\MaNishtanaCziii} \\ \raggedright 
	\vspace*{2ex}
	\MaNishtanaHebiiib\\ 
	\vspace*{1mm}
	\textit{\MaNishtanaTlitiiib} \\{\color{midblue} \MaNishtanaEngiiib} \\ {\color{black} \MaNishtanaCziiib} \\
	\raggedright 
	\par
	\\ \centering
	\LARGE 4.\\ \raggedleft 
	\vspace*{1mm}
	\MaNishtanaHebiv\\
	\vspace*{1mm}
	\textit{\MaNishtanaTlitiv} \\ {\color{midblue} \MaNishtanaEngiv} \\ {\color{black}\MaNishtanaCziv} \\ \raggedright 
	\vspace*{2ex}
	\MaNishtanaHebivb \\ 
	\vspace*{1mm}
	\textit{\MaNishtanaTlitivb} \\{\color{midblue} \MaNishtanaEngivb} \\ {\color{black} \MaNishtanaCzivb} \\
	\par
	}}
	\\
	\vspace{2ex}
	\begin{multicols}{2}
		\MaNishtanaAnswerCz
		
		\columnbreak \MaNishtanaAnswerEng
	\end{multicols}
}

\newcommand{\ArbaAh}{
	The Talmud describes four types of children at the Pesach table, and how to respond to them. \\
	Ty čtyři děti z Talmudu
	\begin{enumerate}
		\item \raggedright{\color{midblue} \childi \\ \color{black} \childicz \\}
		\vspace*{2ex}
		\centering{\color{midblue}\childiq \\ \color{black} \childiqcz} \\
		\vspace{2ex}
		\raggedleft{\color{midblue}{\childia \\ \color{black} \childiacz}}
		
		\item \raggedright{\color{midblue} \childii \\ \color{black} \childiicz \\}
		\vspace*{2ex}
		\centering{\color{midblue}\childiiq \\ \color{black} \childiiqcz} \\
		\vspace{2ex}
		\raggedleft{\color{midblue}{\childiia \\ \color{black} \childiiacz}}
		
		\item \raggedright{\color{midblue} \childiii \\ \color{black} \childiiicz \\}
		\vspace*{2ex}
		\centering{\color{midblue}\childiiiq \\ \color{black} \childiiiqcz} \\
		\vspace{2ex}
		\raggedleft{\color{midblue}{\childiiia \\ \color{black} \childiiiacz}}
		
		\item \raggedright{\color{midblue} \childiv \\ \color{black} \childivcz \\}
		\vspace*{2ex}
		\raggedleft{\color{midblue}{\childiva \\ \color{black} \childivacz}}
		
	\end{enumerate}
}

\newcommand{\MaggidIntro}{%
\begin{Center}
	\textit{\color{black} Naplňte svůj pohár, ale ještě ho nepijte \\ \color{midblue} Pour the second glass. Do not drink it yet.}\\
	\end{Center}
	{\onehalfspacing \color{black} \MaggidText}
	\begin{center}
		\textit{Zvedni svůj pohár: \\ \color{midblue} Raise the glass and read together:}\\
	\vspace*{2ex}
	\color{black}{\Huge \PromiseToastHeb \\}
	%\vspace*{1mm}
	\textit{\PromiseToastTlit}
	\end{center}
	\begin{multicols}{2}
		\PromiseToastCz
		
		\columnbreak 
		\PromiseToastEng
	\end{multicols}
	\centering\textit{now put the glass down without drinking}\\
}
\newcommand{\Plagues}{
	\noindent\color{midblue} G-d brought about 10 Plagues upon the Egyptians in Egypt:\\
	\color{black} Deset ran přivedl Buh –  na Egypťany v Egyptě:\\
	\color{black}\textit{Když čteme název každého moru, dejte na stránku kapku nápoje, jako slzu za utrpení, které ve světě způsobili. \\ \color{midblue}As we read the name of each plague we spill a drop of drink on each one, like a tear for the suffering they caused in the world.}
	\vspace*{2ex}
	\begin{adjustwidth}{15mm}{15mm}
		\begin{multicols}{3}
			\color{black} \PlaguesTextCz
		
			\columnbreak 
			\color{midblue} \PlaguesTextEng
			
			\columnbreak
			\color{darkblue} \PlaguesTextHeb
		\end{multicols}
	\end{adjustwidth}
}

\newcommand{\Dayyeinu}{
	%\textit{
	%	Persian Jews hit each other with a Spring Onion at each of the following lines, to symbolise the whips of the slavemasters.\\
	%	{\large Sarah is allergic to onion so is exempt from being whacked}
	%	}
	
	*maybe find a song version of the dayyeinu and put the words out so we can listen and hit when we spot the word dayyeinu*\\
	\DayyeinuIntro
	\DayyeinuText
	\vspace*{2ex}}
	
\newcommand{\Gamliel}{
	\vspace*{2ex}
	\RabbiGamliel
	\vspace*{2ex}
	\onecolumn
	\EveryGenerationText
	
	We remember how G-d redeemed our ancestors from Egypt, and enabled us to spend this Pesach Seder together. May we arrive at future holidays in peace and happiness.
	}
%
\newcommand{\NetilatYadayim}{
	\textit{Dále si umyjeme ruce s požehnáním \\ {\color{midblue} pour water three times on your right hand then three times on your left hand and then read:}}\\
	\NetilatYadayimHeb
	\textit{\NetilatYadayimTlit}
	\raggedleft
	\begin{multicols}{2}
		\NetilatYadayimCz
		
		\columnbreak 
		\NetilatYadayimEng
	\end{multicols}
}
\newcommand{\HaMotzi}{
	\textit{We bless the unleavened bread just as we bless any bread at the start of a meal:}
	\HaMotziBlessingHeb
	\textit{\HaMotziBlessingTlit}
	\begin{multicols}{2}
		\HaMotziBlessingCz
		
		\columnbreak \HaMotziBlessingEng
	\end{multicols}
}
\newcommand{\Matzah}{
	\textit{Vypravěč ukáže stolovníkům macesy a říká:\\\color{midblue} The Matzah is displayed as the following passage is read:}\\
		\begin{multicols}{2}
		\color{black} Z jakého důvodu jíme tuto maca? Proto, že než se našim předkům 
		zjevil Svatý, buď požehnán, a vykoupil je, nestačilo jim vykynout těsto. Je 
		přec řečeno: „Z těsta, které vynesli z Egypta a které nezkvasilo, pekli 
		macesové placky, jelikož byli z Egypta vyhnáni a nemohli se zdržovat a 
		neudělali si ani zásoby na cestu.“ \\
		
		\columnbreak Why do we eat this Matzah? Because when our ancestors were rescued from Egypt, there was no time to leaven their dough. It is said: “They baked Matzah cakes from the dough they brought out of Egypt, which was not leavened, because they were driven out of Egypt and could not stay there, and they had not prepared any provisions for the journey.”	
	\end{multicols}
	\textit{\color{midblue} We recite the blessing over Matzah because consuming it is a specific Torah command:}
	\MatzahBlessingHeb
	\textit{\MatzahBlessingTlit}
	\begin{multicols}{2}
	\MatzahBlessingCz
	
	\columnbreak \MatzahBlessingEng
\end{multicols}
}
%
\newcommand{\Maror}{
	{\color{midblue}
 In creating a holiday about the joy of freedom, we turn the story of our bitter
history into a sweet celebration. \\ We recognize this by dipping our bitter horseradish (křen) into the sweet charoset. Try to taste both the flavours at once.\\}
\vspace*{2ex}
\color{black} 
\textit{Vypravěč ukáže maror a říká:\\ \color{midblue} The Maror is displayed and the following is read out loud:}\\
\begin{multicols}{2}
	Proč jíme tuto hořkost (maror)? Proto, že Egypťané ztrpčovali život 
	našich předků v Egyptě, jak je řečeno: „Ztrpčovali jejich život dřinou s 
	hlínou a cihlami a vší tou prací na poli a všemi dalšími pracemi, jimiž jim 
	otročili při nucených pracích.“\\
	
	\columnbreak Why do we eat this bitterness? Because the Egyptians made our ancestors' lives ”bitter with hard labour in clay and brick, and all their work in the field, and all their other work with which they made them slaves in forced labour.”
\end{multicols}

\vspace*{2ex}
\textit{namáčíme náš křen do charosetu a řekněte:}\\
\MarorBlessingHeb
\textit{\MarorBlessingTlit}
\begin{multicols}{2}
	\MarorBlessingCz
	
	\columnbreak \MarorBlessingEng
\end{multicols}
}
%
\newcommand{\Koreich}{
	%\includegraphics{Koreich}
	\KoreichText}

\newcommand{\Bareich}{
	\textit{\color{black} Naplňte svůj pohár, ale ještě ho nepijte \\ \color{midblue} Pour the third glass. Do not drink it yet.} \BareichEng}

\newcommand{\Hallel}{%
  \HallelHeb
  	\begin{multicols}{2}
  		\HallelTlit \\
  		\columnbreak \HallelEng
  	\end{multicols}
}

\newcommand{\Elijah}{
	\ElijahText \\
	\vspace*{2ex}
	\StrangersText \\
	\par \centering \textit{We sing this song before closing the door:}\\
	\vspace*{2ex}
	\EliahuHaNavi}

\newcommand{\Miriam}{
\Miriamreading
\MiriamBlessing}

\newcommand{\Nirtzah}{
	As we conclude our Seder, we think about the following year with all its ups, downs and opportunities, and consider how we can bring sweetness to any bitterness we encounter, how we can give and be held by community, and how we will change and grow while retaining the promises, responsibilities and blessings of being Jewish. And either in a literal or a symbolic sense, where Jerusalem signifies a hope for a warm, loving and peaceful community of friends and family, we all raise a glass and toast together: \par
	\NirtzahBlessing}
% File paths: we don't use symlinks as (a) not all platforms support them, and
% (b) they don't fit nicely with the flow we're using.

%\newcommand{\kyriePath}{../Ordinaries/masses/4/kyrie}
%\newcommand{\gloriaPath}{../Ordinaries/masses/4/gloria}
%\newcommand{\sanctusPath}{../Ordinaries/masses/4/sanctus}
%\newcommand{\agnusPath}{../Ordinaries/masses/4/agnus}
%\newcommand{\itePath}{../Ordinaries/masses/4/ite}
% 
%\newcommand{\creedPath}{../Ordinaries/credo/1/credo}
% 
%%% Local Variables:
%%% mode: latex
%%% TeX-master: "missalette"
%%% End:
